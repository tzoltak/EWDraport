\documentclass[b5paper,10pt,twoside]{book}
\usepackage[table]{xcolor}
\definecolor{lightgray}{gray}{0.9}
\usepackage[T1]{fontenc}
\usepackage{emptypage}
\usepackage[polish]{babel}
\usepackage[utf8]{inputenc}
\usepackage{etoolbox}
\usepackage{lmodern}
\usepackage{indentfirst} 
\usepackage{fancyhdr}
\usepackage{xcolor}
\usepackage{etoolbox} 
\usepackage{tocloft}
\usepackage{graphicx}
\usepackage{epstopdf}
\usepackage{geometry}
\usepackage{makeidx}
\usepackage{tabularx}
\usepackage[font=footnotesize ,labelfont=bf]{caption}
\usepackage[hang,flushmargin]{footmisc}
\usepackage{booktabs}
\usepackage{multirow}
\usepackage{mathtools}
\usepackage{apacite}
\usepackage{natbib}
\usepackage[hidelinks]{hyperref}
% dodanie verbatim umożliwia komentowanie wielu linii
\usepackage{verbatim}
\setlength{\footnotesep}{0pt}
\setlength{\skip\footins}{0pt}
\renewcommand\footnoterule{\vspace*{7pt}\hrule width 2.5cm\vspace*{4.6pt}}
\let\lll\undefined
\usepackage{amssymb}
\usepackage{tabularx}
\newcolumntype{S}{@{\stepcounter{Definition}~} >{\bfseries}l @{~--~}X@{}}
\newcounter{Definition}[subsection]
% wyśrodkowanie komórek w tabelkach: \renewcommand{\tabularxcolumn}[1]{>{\small}m{#1}}
\renewcommand{\labelitemi}{\scriptsize$\square$   }
\setlength{\footnotemargin}{3mm}
%\newgeometry{bottom=1.5in, top=1.4in, footskip=0.7in, headsep=0.43in}
\geometry{papersize={170.4mm,243mm}, bottom=2.02795cm, top=2.02795cm, footskip=0.7cm, headsep=0.3527778cm, left=1.5cm, right=1.5cm, textheight=576.00pt}
\selectlanguage{polish}
\pagestyle{fancy}
\fancyhf{}
\renewcommand{\headrulewidth}{0pt}
\renewcommand{\footrulewidth}{0pt}
\renewcommand{\arraystretch}{1.4}\addtolength{\tabcolsep}{-1pt}
\fancyhead{} % clear all fields
\fancyfoot{} % clear all fields
\fancyhead[LE]{Księga wizualizacji składu książki}
\fancyhead[RO]{\nouppercase{\leftmark}}
\fancyfoot[LE,RO]{\thepage}
\makeatletter
\patchcmd{\@fancyhead}{\rlap}{\color{darkgray}\rlap}{}{}
\patchcmd{\headrule}{\hrule}{\color{darkgray}\hrule}{}{}
\makeatother
\widowpenalty=10000
\clubpenalty=10000
\usepackage{lettrine}\sloppy %zakaz wydłużania lini (gdzy nie może złożyć)
\fancypagestyle{plain}{%
   	\fancyhf{}                          % clear all header and footer fields
    \renewcommand{\headrulewidth}{0pt}
	\renewcommand{\footrulewidth}{0pt}
	\fancyfoot[LE,RO]{\thepage}
}
\graphicspath{ {Obrazy/} }
\makeindex
\usepackage[cam,a4,center]{crop}
\parskip 0pt
\usepackage[compact]{titlesec}
\titlespacing{\section}{0pt}{30pt}{15.6pt}
\titlespacing{\subsection}{0pt}{24pt}{12pt}
\titlespacing{\subsubsection}{0pt}{18pt}{8pt}
\titlespacing{\paragraph}{15pt}{2pt}{6pt}
\usepackage{enumitem}
\setlist{nolistsep}
\setlist[itemize]{topsep=0pt}
\renewcommand{\abovecaptionskip}{12pt}
\renewcommand{\belowcaptionskip}{0pt}
\renewcommand{\textfloatsep}{12pt}
\makeatletter
\newcommand\thefontsize{The current font size is: \f@size pt}
\makeatother
\makeatletter
\def\@makechapterhead#1{%
  %%%%\vspace*{50\p@}% %%% removed!
  {\parindent \z@ \raggedright \normalfont
    \ifnum \c@secnumdepth >\m@ne
        \huge\bfseries \@chapapp\space \thechapter
        \par\nobreak
        \vskip 15.2\p@
    \fi
    \interlinepenalty\@M
    \Huge \bfseries #1\par\nobreak
    \vskip 36\p@
  }}
\def\@makeschapterhead#1{%
  %%%%%\vspace*{50\p@}% %%% removed!
  {\parindent \z@ \raggedright
    \normalfont
    \interlinepenalty\@M
    \Huge \bfseries  #1\par\nobreak
    \vskip 36\p@
  }}
\newcommand{\customchapter}[2]{
    \setcounter{chapter}{#1}
    \setcounter{section}{0}
    \chapter*{#2}
    \addcontentsline{toc}{chapter}{#2}
}
\newcommand{\customtitle}[2]{
	\begin{flushleft}\textbf{#1 } #2 \end{flushleft}
	\par\medskip
	\rowcolors{1}{}{lightgray}
}
\makeatother